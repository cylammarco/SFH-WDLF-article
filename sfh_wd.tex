% mnras_template.tex 
%
% LaTeX template for creating an MNRAS paper
%
% v3.0 released 14 May 2015
% (version numbers match those of mnras.cls)
%
% Copyright (C) Royal Astronomical Society 2015
% Authors:
% Keith T. Smith (Royal Astronomical Society)

% Change log
%
% v3.0 May 2015
%    Renamed to match the new package name
%    Version number matches mnras.cls
%    A few minor tweaks to wording
% v1.0 September 2013
%    Beta testing only - never publicly released
%    First version: a simple (ish) template for creating an MNRAS paper

%%%%%%%%%%%%%%%%%%%%%%%%%%%%%%%%%%%%%%%%%%%%%%%%%%
% Basic setup. Most papers should leave these options alone.
\documentclass[fleqn,usenatbib]{mnras}

% MNRAS is set in Times font. If you don't have this installed (most LaTeX
% installations will be fine) or prefer the old Computer Modern fonts, comment
% out the following line
\usepackage{newtxtext,newtxmath}
% Depending on your LaTeX fonts installation, you might get better results with one of these:
%\usepackage{mathptmx}
%\usepackage{txfonts}

% Use vector fonts, so it zooms properly in on-screen viewing software
% Don't change these lines unless you know what you are doing
\usepackage[T1]{fontenc}

% Allow "Thomas van Noord" and "Simon de Laguarde" and alike to be sorted by "N" and "L" etc. in the bibliography.
% Write the name in the bibliography as "\VAN{Noord}{Van}{van} Noord, Thomas"
\DeclareRobustCommand{\VAN}[3]{#2}
\let\VANthebibliography\thebibliography
\def\thebibliography{\DeclareRobustCommand{\VAN}[3]{##3}\VANthebibliography}


%%%%% AUTHORS - PLACE YOUR OWN PACKAGES HERE %%%%%

% Only include extra packages if you really need them. Common packages are:
\usepackage{graphicx}	% Including figure files
\usepackage{amsmath}	% Advanced maths commands
% \usepackage{amssymb}	% Extra maths symbols

%%%%%%%%%%%%%%%%%%%%%%%%%%%%%%%%%%%%%%%%%%%%%%%%%%

%%%%% AUTHORS - PLACE YOUR OWN COMMANDS HERE %%%%%
\usepackage{svg}

% Please keep new commands to a minimum, and use \newcommand not \def to avoid
% overwriting existing commands. Example:
%\newcommand{\pcm}{\,cm$^{-2}$}	% per cm-squared
\newcommand{\msun}{\mathcal{M}_{\sun}}
\newcommand{\orcid}[1]{\href{https://orcid.org/#1}{\includesvg[width=10pt]{orcid}}}

%%%%%%%%%%%%%%%%%%%%%%%%%%%%%%%%%%%%%%%%%%%%%%%%%%

%%%%%%%%%%%%%%%%%%% TITLE PAGE %%%%%%%%%%%%%%%%%%%

% Title of the paper, and the short title which is used in the headers.
% Keep the title short and informative.
\title[Galactic SFH from Gaia eDR3 WDLFs]{First Glimpse of the Galactic Star Formation History from the Gaia eDR3 White Dwarf Luminosity Functions}

% The list of authors, and the short list which is used in the headers.
% If you need two or more lines of authors, add an extra line using \newauthor
\author[M. C. Lam et al.]{
M. C. Lam$^{1\,\orcid{0000-0002-9347-2298}}$\thanks{Contact e-mail: \href{mailto:lam@tau.ac.il}{lam@tau.ac.il}}
\\
% List of institutions
$^{1}$School of Physics and Astronomy, Tel Aviv University, Tel Aviv, Israel 69978
}

% These dates will be filled out by the publisher
\date{Accepted XXX. Received YYY; in original form ZZZ}

% Enter the current year, for the copyright statements etc.
\pubyear{2022}

% Don't change these lines
\begin{document}
\label{firstpage}
\pagerange{\pageref{firstpage}--\pageref{lastpage}}
\maketitle

% Abstract of the paper
\begin{abstract}


\end{abstract}

% Select between one and six entries from the list of approved keywords.
% Don't make up new ones.
\begin{keywords}
keyword1 -- keyword2 -- keyword3
\end{keywords}

%%%%%%%%%%%%%%%%%%%%%%%%%%%%%%%%%%%%%%%%%%%%%%%%%%

%%%%%%%%%%%%%%%%% BODY OF PAPER %%%%%%%%%%%%%%%%%%

\section{Introduction}
%%%%%%%%%%%%%%%%%%%%%%%%%%%%%%%%%%%%%%%%%%%%%%%%%%%%%%%%%%%%%%%%%%%%%%%%%%%%%%%%
White dwarfs~(WDs) are the final stage of stellar evolution of main
sequence~(MS) stars with zero-age MS~(ZAMS) mass less than $8\msun$. Since this
mass range encompasses the vast majority of stars in the Galaxy, these
degenerate remnants are the most common final product of stellar evolution,
thus providing a good sample to study the history of stellar evolution and star
formation in the Galaxy. In this state, there is little nuclear burning to
replenish the energy they radiate away. As a consequence, the luminosity and
temperature decrease monotonically with time. The electron degenerate nature
means that a WD with a typical mass of $0.6\mathcal{M}_{\sun}$ has a similar
size to the Earth, giving rise to their high densities, low luminosities, and
large surface gravities.

%%%%%%%%%%%%%%%%%%%%%%%%%%%%%%%%%%%%%%%%%%%%%%%%%%%%%%%%%%%%%%%%%%%%%%%%%%%%%%%%
The use of the white dwarf luminosity function~(WDLF) as a cosmochronometer was
first introduced by \citet{1959ApJ...129..243S}. Given the finite age of the
Galaxy, there is a minimum temperature below which no white dwarfs can reach in
a limited cooling time. This limit translates to an abrupt downturn in the WDLF
at faint magnitudes. Evidence of such behaviour was observed by
\citet{1979ApJ...233..226L}, however, it was not clear at the time whether it
was due to incompleteness in the observations or to some defect in the
theory~(e.g.,~\citealp{1984ApJ...282..615I}). A decade later,
\citet{1987ApJ...315L..77W} gathered concrete evidence for the downturn and
estimated the age\footnote{``Age'' refers to the total time since the oldest
WD progenitor arrived at the zero-age main sequence.} of the disc to be
$9.3 \pm 2.0$\,Gyr~(see also \citealt{1988ApJ...332..891L}). While most studies
focused on the Galactic discs~\citep{1989LNP...328...15L, 1992ApJ...386..539W,
1995LNP...443...24O, 1998ApJ...497..294L, 1999MNRAS.306..736K,
2012ApJS..199...29G, 2021A&A...649A...6G}, some worked with the stellar
halo~\citep{2006AJ....131..571H, 2011MNRAS.417...93R, 2017AJ....153...10M,
2019MNRAS.482..715L}. 
 
%%%%%%%%%%%%%%%%%%%%%%%%%%%%%%%%%%%%%%%%%%%%%%%%%%%%%%%%%%%%%%%%%%%%%%%%%%%%%%%%
Most WDs have similar broadband colour to main sequence stars, they cannot be
identified using photometry alone. They are found from UV-excess, large
proper motion and/or parallax. Because of the strongly peaked surface gravity
distribution of WDs, photometric fitting for their intrinsic properties
is possible by assuming a surface gravity. WDs fitted in such a way are useful
statistically provided that the sample is not strongly biased. This is
demonstrated in various studies comparing photometric and spectroscopic
solutions to calibrate the atmosphere
model~\citep{2019ApJ...871..169G, 2019ApJ...882..106G}, as well as from the
agreeing shapes of the WDLFs from spectroscopic and photometric samples. The
Gaia satellite provides parallactic measurements for over a billion point
sources~\citep{2021A&A...649A...1G, 2021AJ....161..147B} of which $359,000$
are high confidence WD candidates~\citep[][hereafter, GF21]{2021MNRAS.508.3877G}.
The availability of parallaxes allows much more accurate fitting, particularly
without knowing the surface gravity for the photometric sample. This has
completely  revolutionized the field of WD sciences. In the forthcoming decade,
the Simonyi Survey Telescope at the Vera C. Rubin Observatory will continue to
discover more WDs at fainter magnitudes, but only accompanied by proper
motion measurement at best. Furthermore, at those magnitudes, it is infeasible
to collect spectrum for most of them and thus studies will mostly rely on
photometric methods.

\section{White Dwarf Luminosity Function}
%%%%%%%%%%%%%%%%%%%%%%%%%%%%%%%%%%%%%%%%%%%%%%%%%%%%%%%%%%%%%%%%%%%%%%%%%%%%%%%%
WDLF is a common tool for deriving the age of a stellar population. A WDLF is
the number density of WD as a function of luminosity, it is an evolving
function with time. Its shape and normalisation are determined from only a few
parameters. \citet{1987ApJ...315L..77W} compared an observed WDLF derived from
the Luyten Half-Second~(LHS) catalogue with a theoretical WDLF to obtain an
estimate of the age of the Galaxy for the first time with this technique.
\citet{1990ApJ...352..605N} examined WDLFs with various SFH scenarios. They
showed that WDLF is a sensitive probe of the star formation history~(SFH) as
it shows signatures of irregularities in the SFH such as bursts and lulls.
\citet{2013MNRAS.434.1549R} took it further to address this inverse problem
mathematically and showed some success in recovering the SFH of the solar
neighbourhood when compared against SFH computed from other methods. By
decomposing the disks and halo components of the Milky Way, we can have an
independent view of the past star formation history revealed by only the
WD populations, where they are most useful in deriving the SFH of old
stellar populations~\citep{2011MNRAS.417...93R, 2017ASPC..509...25L}.


%%%%%%%%%%%%%%%%%%%%%%%%%%%%%%%%%%%%%%%%%%%%%%%%%%%%%%%%%%%%%%%%%%%%%%%%%%%%%%%%
The mathematical construction of a WDLF is straightforward: stars
were formed in a distribution of mass~($\mathcal{M}_i$), described by the initial
mass function~(IMF, $\phi$). Then, they spend their lifetime carrying out
nuclear burning~($t_{\mathrm{MS}}$), and the time they spend depends mainly on
their mass. Towards the end stage of stellar evolution stars shed most of the
atmosphere, which is modelled by the initial-final mass relation~(IFMR,
$\zeta$). Once they have become WDs, all that is left is to know how long it has
been cooling~($t_{\mathrm{cool}}$) in order to reach the current
luminosity~($M_\mathrm{bol}$). The heavy-duty of these computations
are coming from interpolation of pre-computed lookup tables. The important
part of this work is to carefully interpolate and integrate over the model
grids, because they are both susceptible to significant rounding errors given
the huge dynamic ranges the variables cover. For example, in the case of a simple
starburst of $\mathcal{O}(10^6)$\,yrs, it requires a relative error
tolerance of $10^{-10}$ in order to integrate properly for an old population.

%%%%%%%%%%%%%%%%%%%%%%%%%%%%%%%%%%%%%%%%%%%%%%%%%%%%%%%%%%%%%%%%%%%%%%%%%%%%%%%%
The integral for a WDLF when parameterised with bolometric magnitude (as
opposed to luminosity) can be written as

\begin{equation}
    n(M_{\mathrm{bol}}) = \int_{\mathcal{M}_l}^{\mathcal{M}_u}
        \tau(M_\mathrm{bol}, \mathcal{M}_f)
        \psi(T_0, M_\mathrm{bol}, \mathcal{M}_i, m, Z)
        \phi(\mathcal{M}_i) d\mathcal{M}_i
\end{equation}
where $n$ is the number density, $\tau$ is the inverse cooling rate, $\psi$ is
the relative star formation rate, $\phi$ is the initial mass function; and their
dependent variables: $M_\mathrm{bol}$ is the absolute bolometric
magnitude, $\mathcal{M}_f$ is the WD mass, $T_0$ is the look-back time, $\mathcal{M}_i$ is
the progenitor MS mass, $Z$ is the metallicity, $\mathcal{M}_l$ is the minimum
progenitor MS mass that could have singly evolved into a WD in the given time,
and $\mathcal{M}_u$ is the maximum progenitor MS mass.

%%%%%%%%%%%%%%%%%%%%%%%%%%%%%%%%%%%%%%%%%%%%%%%%%%%%%%%%%%%%%%%%%%%%%%%%%%%%%%%%
The inverse cooling rate
\begin{equation}
    \tau(M_\mathrm{bol}, \mathcal{M}_f) = \dfrac{dt_{\mathrm{cool}}}{dM_\mathrm{bol}} \left( M_\mathrm{bol}, \mathcal{M}_f \right)
\end{equation}
is a quantity taken from the pre-computed grid of cooling models. 

The relative star formation rate is expressed as a function of look-back time,
\begin{align}
    &\psi(T_0, M_\mathrm{bol}, \mathcal{M}_i, \mathcal{M}_f, Z) =\\
    &\qquad\psi\left[T_0 - t_{\mathrm{cool}}\left(M_\mathrm{bol}, \mathcal{M}_f\right) - t_{\mathrm{MS}}\left(\mathcal{M}_i, Z\right)\right].
\end{align}
The absolute normalisation is not needed when the total stellar mass is coming
from observations; the theoretical WDLF only needs to multiply with a
constant (the total number density) to account for the normalisation.

%%%%%%%%%%%%%%%%%%%%%%%%%%%%%%%%%%%%%%%%%%%%%%%%%%%%%%%%%%%%%%%%%%%%%%%%%%%%%%%%
The IFMR takes a simple form of
\begin{equation}
    \mathcal{M}_f = \zeta(\mathcal{M}_i),
\end{equation}
although there is evidence that more metal-rich stars lose more
envelope~\citep{2007ApJ...671..761K}, there is insufficient empirical data to
derive an IFMR at metallicity much lower or higher than solar abundance.

\section{Retrieving Star Formation History from a WD population}

\citep{1990ApJ...352..605N}
\citep{1992ApJ...386..539W}

\subsection{Spectroscopic Volume Complete Sample}

\citep{2014ApJ...791...92T}
\citep{2019ApJ...878L..11I}


\subsection{WDLF Inversion of a Statistical Sample}
With the inversion algorithm, it is possible to resolve the SFH in high time resolution~\citep{2013MNRAS.434.1549R}. However, it is prone to amplify noise into enhanced star formation upon the inversion~\citep{2014ApJ...791...92T}. This is due to the large number of degenerate solutions that can
yield the WDLF agree to within the uncertainty.


\subsection{Forward modelling of WDLFs}



\section{A New method}
Historically, all the luminosity funtions of the solar neighbourhood were reported
in the bolometric magnitudes. Essentially, all works were reporting the
WD-Bolometric-LF. To address the problem of degeneracy in the solution, we explore
the use of colour information from the luminosity function, where we are using
the WDLF in multiple filters . 


\subsection{Partial WDLFs}
Inspired by the partial CMD designed in~\citep{2006A&A...459..783C}, we use partial WDLFs as building
blocks of the fitting models.


Testing
\subsection{Model Fitting}

\subsubsection{Bolometric Luminosity/Magnitude}
The likelihood function to be maximized is essentially minimising the $\chi^2$ between
the observed and the measured WDLFs weighted by the variance.

\begin{figure}
    \centering
    \includegraphics[width=\columnwidth]{test_case/two_bursts_sfh_with_noise.png}
    \caption{Caption}
    \label{fig:two_bursts_with_noise}
\end{figure}

When the function is properly smoothed and weighted, the parametrisation with luminosity and magnitude should give identical results.


\subsection{Effect of magnitude bin size}
\begin{figure}
    \centering
    \includegraphics[width=\columnwidth]{test_case/two_bursts_sfh_compare_bin_size.png}
    \caption{Caption}
    \label{fig:two_burst_compare_bin}
\end{figure}

\subsection{Effect of time bin size}


\subsection{Effect of choice of models}



\section{Application to the early Gaia Data Release 3}
\begin{figure}
    \centering
    \includegraphics[width=\columnwidth]{test_case/gcns_sfh.png}
    \includegraphics[width=\columnwidth]{test_case/gcns_reconstructed_wdlf.png}
    \caption{Caption}
    \label{fig:gcns_mbol}
\end{figure}



\subsection{Comparison with WDLF inversions}



\subsection{Comparison with previous studies}



\section{Conclusions}



\section*{Acknowledgements}


%%%%%%%%%%%%%%%%%%%%%%%%%%%%%%%%%%%%%%%%%%%%%%%%%%
\section*{Data Availability}


%%%%%%%%%%%%%%%%%%%% REFERENCES %%%%%%%%%%%%%%%%%%

% The best way to enter references is to use BibTeX:

\bibliographystyle{mnras}
\bibliography{sfh_wd} % if your bibtex file is called example.bib


% Alternatively you could enter them by hand, like this:
% This method is tedious and prone to error if you have lots of references
%\begin{thebibliography}{99}
%\bibitem[\protect\citeauthoryear{Author}{2012}]{Author2012}
%Author A.~N., 2013, Journal of Improbable Astronomy, 1, 1
%\bibitem[\protect\citeauthoryear{Others}{2013}]{Others2013}
%Others S., 2012, Journal of Interesting Stuff, 17, 198
%\end{thebibliography}

%%%%%%%%%%%%%%%%%%%%%%%%%%%%%%%%%%%%%%%%%%%%%%%%%%

%%%%%%%%%%%%%%%%% APPENDICES %%%%%%%%%%%%%%%%%%%%%

\appendix

\section{Some extra material}

%%%%%%%%%%%%%%%%%%%%%%%%%%%%%%%%%%%%%%%%%%%%%%%%%%


% Don't change these lines
\bsp	% typesetting comment
\label{lastpage}
\end{document}

% End of mnras_template.tex
